\documentclass[12pt,a4paper,oneside]{article}

\usepackage[QX]{polski}

\usepackage[utf8]{inputenc}
\usepackage{latexsym}
\usepackage{tgpagella}
\usepackage{lmodern}
\usepackage{amsmath,amsthm,amsfonts,amssymb,alltt}
\usepackage{epsfig}
\usepackage{pdflscape}
\usepackage{caption}
\usepackage{indentfirst}
\usepackage{float}
%\usepackage{showkeys}
\bibliographystyle{plabbrv}


\usepackage{color}
\usepackage[polish]{babel}
\usepackage{datetime2}
\usepackage[x11names,dvipsnames,table]{xcolor}
\usepackage{hyperref}
\hypersetup{
    pdfauthor={Roman Czapla, Olaf Bar},
    colorlinks=True,
    linkcolor=darkgray,  % color of internal links (change box color with linkbordercolor)
    citecolor=BrickRed,  % color of links to bibliography
    filecolor=Magenta,   % color of file links
    urlcolor=BlueViolet}	%%pdfpagemode=FullScreen}

% diagramy, grafy itp.
\usepackage{tikz}
\usetikzlibrary{positioning}
\usetikzlibrary{arrows}
\usetikzlibrary{arrows.meta}
\usetikzlibrary{chains,fit,shapes,calc}
\tikzset{main node/.style={circle,fill=blue!20,draw,minimum size=1cm,inner sep=0pt}}

% algorytmy
\usepackage[linesnumbered,lined,commentsnumbered]{algorithm2e}
\SetKwFor{ForEach}{for each}{do}{end for}%
\SetKwFor{ForAll}{for all}{do}{end for}%
\newenvironment{myalgorithm}
{\rule{\textwidth}{0.5mm}\\\SetAlCapSty{}\SetAlgoNoEnd\SetAlgoNoLine\begin{algorithm}}{\end{algorithm}\rule{\textwidth}{0.5mm}}


%---------------------
\overfullrule=2mm
\pagestyle{plain}
\textwidth=15cm \textheight=685pt \topmargin=-25pt \linespread{1.3}
\setlength{\parskip}{0pt}
\setlength\arraycolsep{2pt}
\oddsidemargin = 0.9cm
\evensidemargin =-0.1cm

\captionsetup{width=.95\linewidth, justification=centering}
%---------------------




\newtheorem{tw}{Twierdzenie}[section]
\newtheorem{lem}[tw]{Lemat}
\newtheorem{co}[tw]{Wniosek}
\newtheorem{prop}[tw]{Stwierdzenie}
\theoremstyle{definition}
\newtheorem{ex}{Przykład}
\newtheorem{re}[tw]{Uwaga}
\newtheorem{de}{Definicja}[section]



\newcommand{\bC}{{\mathbb C}}
\newcommand{\bR}{{\mathbb R}}
\newcommand{\bZ}{{\mathbb Z}}
\newcommand{\bQ}{{\mathbb Q}}
\newcommand{\bN}{{\mathbb N}}
\newcommand{\captionT}[1]{\caption{\textsc{\footnotesize{#1}}}}
\renewcommand\figurename{Rys.}

\numberwithin{equation}{section}
\renewcommand{\thefootnote}{\arabic{footnote})}
%\renewcommand{\thefootnote}{\alph{footnote})}



\begin{document}

% --------------------------------------------
% Strona tytułowa
% --------------------------------------------

    \thispagestyle{empty}
    \begin{titlepage}
        \begin{center}\Large
        Uniwersytet Komisji Edukacji Narodowej w Krakowie\\
        \large
        Instytut Bezpieczeństwa i Informatyki\\
        \vskip 10pt
        \end{center}
        \begin{center}
            \centering \includegraphics[width=1.0\columnwidth]{images/logo.png}
        \end{center}

        \begin{center}
        {\bf \fontsize{14pt}{14pt}\selectfont PROJEKT INŻYNIERSKI \\ DOKUMENTACJA PROJEKTOWA}
        \end{center}
        \vskip 5pt
        \begin{center}
        {\bf \fontsize{22pt}{22pt}\selectfont Serwis Ogłoszeniowy BINM}
        \end{center}



        \begin{center}
        {\fontsize{12pt}{12pt}\selectfont wykonany przez: }
        \end{center}
        \begin{center}
        {\bf\fontsize{16pt}{16pt}\selectfont Jakub ...}\\
        {\fontsize{12pt}{12pt}\selectfont Nr albumu: XXXXX \\\&\\}
        {\bf\fontsize{16pt}{16pt}\selectfont Imię Nazwisko (Frontend)}\\
        {\fontsize{12pt}{12pt}\selectfont Nr albumu: XXXXX\\\&\\}
        {\bf\fontsize{16pt}{16pt}\selectfont Imię Nazwisko (Mobile)}\\
        {\fontsize{12pt}{12pt}\selectfont Nr albumu: XXXXX}
        \end{center}
        \begin{center}
        {\fontsize{12pt}{12pt}\selectfont pod opieką:}\\
        {\bf\fontsize{12pt}{12pt}\selectfont dr inż. ...}
        \end{center}

%\mbox{}
        \vspace*{\fill}
%\vskip 50pt
        \begin{center}
            \large
            Kraków \the\year\\
            (ostatnia aktualizacja: \DTMcurrenttime,\;\today)
        \end{center}
    \end{titlepage}
    \setcounter{page}{0}
    \newpage\null\thispagestyle{empty}
%\setcounter{page}{0} 
%\newpage
%\thispagestyle{empty}

    \tableofcontents


    \newpage

    \section{Szczegółowa dokumentacja projektowa}
    \textit{W zależności od specyfiki projektu! Wymienione niżej podpunkty mają charakter orientacyjny.}
    \subsection{Projekt UML}
    \textit{W szczególności: diagram klas, ew. np. przypadki użycia, diagramy sekwencji, czynności, stanów, obiektów/komponentów/pakietów itp.}

    \subsection{Projekt bazy danych}
    Projekt wykorzystuje relacyjną bazę danych PostgreSQL. Schemat bazy danych został zaprojektowany w podejściu \textit{Code-First} przy użyciu JPA (Java Persistence API) i Hibernate. Poniżej znajduje się opis kluczowych tabel.

    \begin{center}
        \centering \includegraphics[width=1.0\columnwidth]{dokumentacja_projektowa/images/Diagram ERD.png}
    \end{center}

    \subsubsection{Tabela \texttt{tbl\_users}}
    Przechowuje informacje o użytkownikach systemu.
    \begin{table}[H]
        \centering
        \begin{tabular}{|l|l|l|}
            \hline
            \textbf{Nazwa kolumny} & \textbf{Typ danych} & \textbf{Opis} \\
            \hline
            \texttt{id} & BIGINT (PK) & Klucz główny, auto-inkrementacja \\
            \texttt{user\_id} & VARCHAR(255) & Publiczny, unikalny identyfikator (UUID) \\
            \texttt{name} & VARCHAR(255) & Nazwa użytkownika \\
            \texttt{email} & VARCHAR(255) & Unikalny adres email \\
            \texttt{password} & VARCHAR(255) & Hasło (zahaszowane algorytmem BCrypt) \\
            \texttt{profile\_image\_url} & VARCHAR(255) & URL do zdjęcia profilowego \\
            \texttt{role} & VARCHAR(255) & Rola użytkownika (USER, ADMIN) \\
            \texttt{is\_account\_verified} & BOOLEAN & Czy konto zostało zweryfikowane przez OTP \\
            \texttt{verify\_otp} & VARCHAR(255) & Kod weryfikacyjny OTP \\
            \texttt{verify\_otp\_expire\_at} & BIGINT & Czas wygaśnięcia kodu weryfikacyjnego \\
            \texttt{reset\_otp} & VARCHAR(255) & Kod OTP do resetu hasła \\
            \texttt{reset\_otp\_expire\_at} & BIGINT & Czas wygaśnięcia kodu resetu hasła \\
            \texttt{created\_at} & TIMESTAMP & Data utworzenia konta \\
            \texttt{updated\_at} & TIMESTAMP & Data ostatniej modyfikacji \\
            \hline
        \end{tabular}
        \caption{Struktura tabeli \texttt{tbl\_users}}
    \end{table}

    \subsubsection{Tabela \texttt{listing}}
    Główna tabela przechowująca ogłoszenia.
    \begin{table}[H]
        \centering
        \begin{tabular}{|l|l|l|}
            \hline
            \textbf{Nazwa kolumny} & \textbf{Typ danych} & \textbf{Opis} \\
            \hline
            \texttt{id} & BIGINT (PK) & Klucz główny \\
            \texttt{public\_id} & UUID & Publiczny, unikalny identyfikator \\
            \texttt{seller\_user\_id} & VARCHAR(255) & ID sprzedawcy (FK do \texttt{tbl\_users}) \\
            \texttt{category\_id} & BIGINT (FK) & ID kategorii (FK do \texttt{category}) \\
            \texttt{title} & VARCHAR(255) & Tytuł ogłoszenia \\
            \texttt{description} & TEXT & Opis ogłoszenia \\
            \texttt{price\_amount} & NUMERIC(12, 2) & Cena \\
            \texttt{currency} & VARCHAR(3) & Waluta (np. PLN) \\
            \texttt{negotiable} & BOOLEAN & Czy cena podlega negocjacji \\
            \texttt{location\_city} & VARCHAR(255) & Miasto \\
            \texttt{location\_region} & VARCHAR(255) & Region/Województwo \\
            \texttt{latitude} & DOUBLE PRECISION & Szerokość geograficzna \\
            \texttt{longitude} & DOUBLE PRECISION & Długość geograficzna \\
            \texttt{contact\_phone\_number} & VARCHAR(255) & Numer telefonu kontaktowego \\
            \texttt{status} & VARCHAR(255) & Status ogłoszenia (DRAFT, WAITING, ACTIVE, etc.) \\
            \texttt{published\_at} & TIMESTAMP & Data publikacji \\
            \texttt{expires\_at} & TIMESTAMP & Data wygaśnięcia \\
            \texttt{created\_at} & TIMESTAMP & Data utworzenia \\
            \texttt{updated\_at} & TIMESTAMP & Data ostatniej modyfikacji \\
            \hline
        \end{tabular}
        \caption{Struktura tabeli \texttt{listing}}
    \end{table}

    \subsubsection{Tabela \texttt{listing\_media}}
    Przechowuje zdjęcia przypisane do ogłoszeń.
    \begin{table}[H]
        \centering
        \begin{tabular}{|l|l|l|}
            \hline
            \textbf{Nazwa kolumny} & \textbf{Typ danych} & \textbf{Opis} \\
            \hline
            \texttt{id} & BIGINT (PK) & Klucz główny \\
            \texttt{listing\_id} & BIGINT (FK) & ID ogłoszenia \\
            \texttt{media\_url} & VARCHAR(255) & URL do pliku w chmurze \\
            \texttt{media\_type} & VARCHAR(50) & Typ pliku (np. image) \\
            \texttt{position} & INTEGER & Kolejność wyświetlania \\
            \hline
        \end{tabular}
        \caption{Struktura tabeli \texttt{listing\_media}}
    \end{table}

    \subsubsection{Tabela \texttt{category}}
    Przechowuje hierarchiczną strukturę kategorii.
    \begin{table}[H]
        \centering
        \begin{tabular}{|l|l|l|}
            \hline
            \textbf{Nazwa kolumny} & \textbf{Typ danych} & \textbf{Opis} \\
            \hline
            \texttt{id} & BIGINT (PK) & Klucz główny \\
            \texttt{parent\_id} & BIGINT (FK) & ID kategorii nadrzędnej (self-reference) \\
            \texttt{name} & VARCHAR(255) & Nazwa kategorii \\
            \texttt{sort\_order} & INTEGER & Kolejność sortowania \\
            \texttt{image\_url} & VARCHAR(255) & Ikona/Obrazek kategorii \\
            \texttt{depth} & INTEGER & Głębokość w drzewie kategorii \\
            \texttt{is\_leaf} & BOOLEAN & Czy jest to kategoria końcowa (liść) \\
            \hline
        \end{tabular}
        \caption{Struktura tabeli \texttt{category}}
    \end{table}

    \subsubsection{Tabela \texttt{attribute\_definition}}
    Definiuje atrybuty (cechy), które mogą posiadać ogłoszenia w danej kategorii.
    \begin{table}[H]
        \centering
        \begin{tabular}{|l|l|l|}
            \hline
            \textbf{Nazwa kolumny} & \textbf{Typ danych} & \textbf{Opis} \\
            \hline
            \texttt{id} & BIGINT (PK) & Klucz główny \\
            \texttt{category\_id} & BIGINT (FK) & ID kategorii \\
            \texttt{key} & VARCHAR(255) & Klucz atrybutu (np. color) \\
            \texttt{label} & VARCHAR(255) & Etykieta wyświetlana (np. Kolor) \\
            \texttt{type} & VARCHAR(50) & Typ danych (STRING, NUMBER, BOOLEAN, ENUM) \\
            \texttt{unit} & VARCHAR(50) & Jednostka (np. km, cm) \\
            \texttt{sort\_order} & INTEGER & Kolejność wyświetlania \\
            \texttt{active} & BOOLEAN & Czy atrybut jest aktywny \\
            \hline
        \end{tabular}
        \caption{Struktura tabeli \texttt{attribute\_definition}}
    \end{table}

    \subsubsection{Tabela \texttt{attribute\_option}}
    Przechowuje możliwe wartości dla atrybutów typu ENUM.
    \begin{table}[H]
        \centering
        \begin{tabular}{|l|l|l|}
            \hline
            \textbf{Nazwa kolumny} & \textbf{Typ danych} & \textbf{Opis} \\
            \hline
            \texttt{id} & BIGINT (PK) & Klucz główny \\
            \texttt{attribute\_id} & BIGINT (FK) & ID definicji atrybutu \\
            \texttt{value} & VARCHAR(255) & Wartość opcji \\
            \texttt{label} & VARCHAR(255) & Etykieta opcji \\
            \texttt{sort\_order} & INTEGER & Kolejność sortowania \\
            \hline
        \end{tabular}
        \caption{Struktura tabeli \texttt{attribute\_option}}
    \end{table}

    \subsubsection{Tabela \texttt{listing\_attribute}}
    Przechowuje konkretne wartości atrybutów dla danego ogłoszenia.
    \begin{table}[H]
        \centering
        \begin{tabular}{|l|l|l|}
            \hline
            \textbf{Nazwa kolumny} & \textbf{Typ danych} & \textbf{Opis} \\
            \hline
            \texttt{id} & BIGINT (PK) & Klucz główny \\
            \texttt{listing\_id} & BIGINT (FK) & ID ogłoszenia \\
            \texttt{attribute\_id} & BIGINT (FK) & ID definicji atrybutu \\
            \texttt{option\_id} & BIGINT (FK) & ID wybranej opcji (dla ENUM) \\
            \texttt{v\_text} & VARCHAR(255) & Wartość tekstowa \\
            \texttt{v\_number} & NUMERIC & Wartość liczbowa \\
            \texttt{v\_boolean} & BOOLEAN & Wartość logiczna \\
            \hline
        \end{tabular}
        \caption{Struktura tabeli \texttt{listing\_attribute}}
    \end{table}

    \subsubsection{Tabela \texttt{favorite}}
    Przechowuje informacje o polubieniach (ulubionych ogłoszeniach).
    \begin{table}[H]
        \centering
        \begin{tabular}{|l|l|l|}
            \hline
            \textbf{Nazwa kolumny} & \textbf{Typ danych} & \textbf{Opis} \\
            \hline
            \texttt{id} & BIGINT (PK) & Klucz główny \\
            \texttt{user\_id} & VARCHAR(255) & ID użytkownika \\
            \texttt{entity\_id} & VARCHAR(255) & ID polubionego obiektu (np. ogłoszenia) \\
            \texttt{entity\_type} & VARCHAR(50) & Typ obiektu (np. LISTING) \\
            \texttt{created\_at} & TIMESTAMP & Data dodania do ulubionych \\
            \hline
        \end{tabular}
        \caption{Struktura tabeli \texttt{favorite}}
    \end{table}

    \subsubsection{Tabela \texttt{conversation}}
    Reprezentuje konwersację między dwoma użytkownikami na temat ogłoszenia.
    \begin{table}[H]
        \centering
        \begin{tabular}{|l|l|l|}
            \hline
            \textbf{Nazwa kolumny} & \textbf{Typ danych} & \textbf{Opis} \\
            \hline
            \texttt{id} & BIGINT (PK) & Klucz główny \\
            \texttt{listing\_id} & UUID & ID ogłoszenia, którego dotyczy rozmowa \\
            \texttt{buyer\_id} & VARCHAR(255) & ID kupującego \\
            \texttt{seller\_id} & VARCHAR(255) & ID sprzedającego \\
            \texttt{created\_at} & TIMESTAMP & Data utworzenia \\
            \texttt{updated\_at} & TIMESTAMP & Data ostatniej wiadomości \\
            \hline
        \end{tabular}
        \caption{Struktura tabeli \texttt{conversation}}
    \end{table}

    \subsubsection{Tabela \texttt{message}}
    Przechowuje pojedynczą wiadomość w ramach konwersacji.
    \begin{table}[H]
        \centering
        \begin{tabular}{|l|l|l|}
            \hline
            \textbf{Nazwa kolumny} & \textbf{Typ danych} & \textbf{Opis} \\
            \hline
            \texttt{id} & BIGINT (PK) & Klucz główny \\
            \texttt{conversation\_id} & BIGINT (FK) & ID konwersacji \\
            \texttt{sender\_id} & VARCHAR(255) & ID nadawcy \\
            \texttt{recipient\_id} & VARCHAR(255) & ID odbiorcy \\
            \texttt{content} & TEXT & Treść wiadomości \\
            \texttt{is\_read} & BOOLEAN & Czy wiadomość została odczytana \\
            \texttt{created\_at} & TIMESTAMP & Data wysłania \\
            \hline
        \end{tabular}
        \caption{Struktura tabeli \texttt{message}}
    \end{table}

    \subsection{Szczegółowa dokumentacja kodu (Backend)}
    Część serwerowa aplikacji została zaimplementowana w języku Java (wersja 21) przy użyciu frameworka Spring Boot. Architektura projektu opiera się na podziale na moduły funkcjonalne (\textit{Package by Feature}), co zapewnia wysoką spójność i niskie sprzężenie komponentów.

    \subsubsection{Architektura i Pakiety}
    Główne pakiety aplikacji to:
    \begin{itemize}
        \item \texttt{com.BINM.user}: Zarządzanie użytkownikami, autentykacja, profile.
        \item \texttt{com.BINM.listing}: Zarządzanie ogłoszeniami, kategoriami i atrybutami.
        \item \texttt{com.BINM.messaging}: System komunikacji w czasie rzeczywistym (WebSocket).
        \item \texttt{com.BINM.interactions}: Obsługa ulubionych i innych interakcji.
        \item \texttt{com.BINM.media}: Obsługa przesyłania plików (Azure Blob Storage).
    \end{itemize}

    \subsubsection{Kluczowe Klasy i Interfejsy}

    \textbf{1. Fasady (Facades)}
    Projekt wykorzystuje wzorzec Fasady, aby ukryć złożoność logiki biznesowej przed kontrolerami i innymi modułami.
    \begin{itemize}
        \item \texttt{ListingFacade}: Główny punkt dostępu do operacji na ogłoszeniach (tworzenie, edycja, wyszukiwanie).
        \item \texttt{ProfileFacade}: Zarządzanie profilem użytkownika i autentykacją.
        \item \texttt{MessagingFacade}: Obsługa wiadomości i konwersacji.
        \item \texttt{InteractionFacade}: Zarządzanie ulubionymi.
    \end{itemize}

    \textbf{2. Serwisy (Services)}
    Implementują logikę biznesową zdefiniowaną w interfejsach fasad.
    \begin{itemize}
        \item \texttt{ListingService}: Odpowiada za cykl życia ogłoszenia (Draft $\rightarrow$ Waiting $\rightarrow$ Active $\rightarrow$ Completed/Expired). Implementuje logikę walidacji i zapisu.
        \item \texttt{SearchService}: Dedykowany serwis do obsługi zaawansowanego wyszukiwania ogłoszeń z wykorzystaniem dynamicznych filtrów.
        \item \texttt{MessagingService}: Obsługuje logikę czatu, tworzenie konwersacji i zapisywanie wiadomości.
    \end{itemize}

    \textbf{3. Kontrolery (Controllers)}
    Wystawiają endpointy REST API. Podzielone są na publiczne (dostępne dla wszystkich), użytkownika (wymagające logowania) i administracyjne (wymagające roli ADMIN).
    \begin{itemize}
        \item \texttt{ListingPublicController}, \texttt{ListingUserController}, \texttt{ListingAdminController}
        \item \texttt{AuthController}, \texttt{ProfileController}
        \item \texttt{ChatController} (WebSocket), \texttt{ConversationController} (REST)
    \end{itemize}

    \subsubsection{Wzorce Projektowe}
    W projekcie zastosowano szereg wzorców projektowych, m.in.:

    \begin{itemize}
        \item \textbf{Facade (Fasada):} Używana do uproszczenia interfejsów modułów i ukrycia szczegółów implementacyjnych (np. \texttt{ListingFacade}).
        \item \textbf{Strategy (Strategia):} Wykorzystywana w mechanizmie wyszukiwania (\texttt{Specification<Listing>}), gdzie kryteria filtrowania są budowane dynamicznie w zależności od żądania użytkownika.
        \item \textbf{Observer / Publish-Subscribe:} Zrealizowany za pomocą Spring Events. Przykład: Gdy ogłoszenie zostaje zakończone (\texttt{ListingFinishedEvent}), moduł interakcji automatycznie usuwa je z ulubionych, nie tworząc bezpośredniej zależności między serwisami.
        \item \textbf{Builder:} Wykorzystywany (dzięki bibliotece Lombok) do tworzenia złożonych obiektów DTO i encji w czytelny sposób.
        \item \texttt{Repository:} Wzorzec dostępu do danych, zaimplementowany przez Spring Data JPA.
    \end{itemize}

    \subsection{Szczegółowa dokumentacja kodu (Frontend)}
    \textit{[Miejsce dla zespołu frontendowego]}

    \subsection{Szczegółowa dokumentacja kodu (Mobile)}
    \textit{[Miejsce dla zespołu mobilnego]}

    \subsection{Środowisko programistyczne}
    Do rozwoju i uruchomienia projektu wykorzystano następujące narzędzia i technologie:

    \begin{itemize}
        \item \textbf{Język programowania:} Java 21.
        \item \textbf{Framework:} Spring Boot 3.x (Spring Web, Spring Security, Spring Data JPA, Spring WebSocket).
        \item \textbf{Baza danych:} PostgreSQL 15+.
        \item \textbf{Zarządzanie zależnościami:} Apache Maven.
        \item \textbf{IDE:} IntelliJ IDEA Ultimate / Community.
        \item \textbf{Konteneryzacja:} Docker (do uruchomienia bazy danych).
        \item \textbf{Przechowywanie plików:} Azure Blob Storage.
        \item \textbf{Biblioteki pomocnicze:} Lombok (redukcja boilerplate code), MapStruct (mapowanie DTO-Entity), JJWT (obsługa tokenów JWT).
    \end{itemize}





    \renewcommand\refname{Literatura (jeżeli wymagana)}
    \bibliography{references}
    \addcontentsline{toc}{section}{Literatura}
% --------------------------------------------------------------------
%%%%%%% odkomentować gdy bibliografia ma być wewnątrz dokumentu
% --------------------------------------------------------------------
%\begin{thebibliography}{11}
%
%\addcontentsline{toc}{section}{Literatura}
%
%\bibitem{ZAN}
%C. Zannoni and P. Pasini, 
%\emph{Advances in the Computer Simulatons of Liquid Crystals}, Kluwer Academic Publishers, 2000.
%
%\end{thebibliography}

\end{document}
