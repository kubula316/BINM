\documentclass[12pt,a4paper,oneside]{article}

\usepackage[QX]{polski}

\usepackage[utf8]{inputenc}
\usepackage{latexsym}
\usepackage{tgpagella}
\usepackage{lmodern}
\usepackage{amsmath,amsthm,amsfonts,amssymb,alltt}
\usepackage{epsfig}
\usepackage{pdflscape}
\usepackage{caption}
\usepackage{indentfirst}
\usepackage{float}
%\usepackage{showkeys}
\bibliographystyle{plabbrv}


\usepackage{color}
\usepackage[polish]{babel}
\usepackage{datetime2}
\usepackage[x11names,dvipsnames,table]{xcolor}
\usepackage{hyperref}
\hypersetup{
pdfauthor={Roman Czapla, Olaf Bar},
colorlinks=True,
linkcolor=darkgray,  % color of internal links (change box color with linkbordercolor)
citecolor=BrickRed,  % color of links to bibliography
filecolor=Magenta,   % color of file links
urlcolor=BlueViolet}	%%pdfpagemode=FullScreen}

% diagramy, grafy itp.
\usepackage{tikz}
\usetikzlibrary{positioning}
\usetikzlibrary{arrows}
\usetikzlibrary{arrows.meta}
\usetikzlibrary{chains,fit,shapes,calc}
\tikzset{main node/.style={circle,fill=blue!20,draw,minimum size=1cm,inner sep=0pt}}

% algorytmy
\usepackage[linesnumbered,lined,commentsnumbered]{algorithm2e}
\SetKwFor{ForEach}{for each}{do}{end for}%
\SetKwFor{ForAll}{for all}{do}{end for}%
\newenvironment{myalgorithm}
{\rule{\textwidth}{0.5mm}\\\SetAlCapSty{}\SetAlgoNoEnd\SetAlgoNoLine\begin{algorithm}}{\end{algorithm}\rule{\textwidth}{0.5mm}}


%---------------------
\overfullrule=2mm
\pagestyle{plain}
\textwidth=15cm \textheight=685pt \topmargin=-25pt \linespread{1.3} 
\setlength{\parskip}{0pt}
\setlength\arraycolsep{2pt}
\oddsidemargin = 0.9cm
\evensidemargin =-0.1cm

\captionsetup{width=.95\linewidth, justification=centering}
%---------------------




\newtheorem{tw}{Twierdzenie}[section]
\newtheorem{lem}[tw]{Lemat}
\newtheorem{co}[tw]{Wniosek}
\newtheorem{prop}[tw]{Stwierdzenie}
\theoremstyle{definition}
\newtheorem{ex}{Przykład}
\newtheorem{re}[tw]{Uwaga}
\newtheorem{de}{Definicja}[section]



\newcommand{\bC}{{\mathbb C}}
\newcommand{\bR}{{\mathbb R}}
\newcommand{\bZ}{{\mathbb Z}}
\newcommand{\bQ}{{\mathbb Q}}
\newcommand{\bN}{{\mathbb N}}
\newcommand{\captionT}[1]{\caption{\textsc{\footnotesize{#1}}}}
\renewcommand\figurename{Rys.}

\numberwithin{equation}{section}
\renewcommand{\thefootnote}{\arabic{footnote})}
%\renewcommand{\thefootnote}{\alph{footnote})}



\begin{document}

% --------------------------------------------
% Strona tytułowa
% --------------------------------------------

\thispagestyle{empty}
\begin{titlepage}
\begin{center}\Large
Uniwersytet Komisji Edukacji Narodowej w Krakowie\\
\large
Instytut Bezpieczeństwa i Informatyki\\
\vskip 10pt
\end{center}
\begin{center}
\centering \includegraphics[width=1.0\columnwidth]{images/logo.png}
\end{center}

\begin{center}
 {\bf \fontsize{14pt}{14pt}\selectfont PROJEKT INŻYNIERSKI \\ DOKUMENTACJA PROJEKTOWA}
\end{center}
\vskip 5pt
\begin{center}
 {\bf \fontsize{22pt}{22pt}\selectfont Serwis Ogłoszeniowy BINM}
\end{center}



\begin{center}
 {\fontsize{12pt}{12pt}\selectfont wykonany przez: }
\end{center}
\begin{center}
 {\bf\fontsize{16pt}{16pt}\selectfont Jakub Siudek}\\
 {\fontsize{12pt}{12pt}\selectfont Nr albumu: 161028 \\\&\\}
 {\bf\fontsize{16pt}{16pt}\selectfont Imię Nazwisko (Frontend)}\\
 {\fontsize{12pt}{12pt}\selectfont Nr albumu: XXXXX\\\&\\}
 {\bf\fontsize{16pt}{16pt}\selectfont Imię Nazwisko (Mobile)}\\
 {\fontsize{12pt}{12pt}\selectfont Nr albumu: XXXXX}
\end{center}
\begin{center}
 {\fontsize{12pt}{12pt}\selectfont pod opieką:}\\
 {\bf\fontsize{12pt}{12pt}\selectfont dr inż. ...}
\end{center}

%\mbox{}
\vspace*{\fill}
%\vskip 50pt
\begin{center}
\large
Kraków \the\year\\
(ostatnia aktualizacja: \DTMcurrenttime,\;\today)
\end{center}
\end{titlepage}
\setcounter{page}{0}
\newpage\null\thispagestyle{empty}
%\setcounter{page}{0}
%\newpage
%\thispagestyle{empty}

\tableofcontents


\newpage

\section{Szczegółowa dokumentacja projektowa}
\textit{W zależności od specyfiki projektu! Wymienione niżej podpunkty mają charakter orientacyjny.}

\subsection{Projekt UML}
System został zaprojektowany w architekturze klient-serwer, wykorzystując model REST API do komunikacji. Głównym elementem systemu jest serwer aplikacji (Backend), który udostępnia interfejsy programistyczne dla aplikacji klienckich (Frontend webowy oraz aplikacja mobilna).

Architektura systemu składa się z następujących komponentów:
\begin{itemize}
    \item \textbf{Aplikacja Webowa (Frontend):} Interfejs użytkownika dostępny przez przeglądarkę internetową, zaimplementowany w technologii React.
    \item \textbf{Aplikacja Mobilna:} Natywna aplikacja na system Android, umożliwiająca korzystanie z serwisu na urządzeniach przenośnych.
    \item \textbf{Backend API:} Serwer aplikacji zaimplementowany w języku Java (Spring Boot), odpowiedzialny za logikę biznesową, autentykację, obsługę ogłoszeń oraz komunikację w czasie rzeczywistym.
    \item \textbf{Baza Danych:} Relacyjna baza danych PostgreSQL przechowująca wszystkie trwałe dane systemu.
    \item \textbf{Usługi Zewnętrzne:}
    \begin{itemize}
        \item \textbf{Serwer SMTP:} Do wysyłania wiadomości email (weryfikacja konta, reset hasła).
        \item \textbf{Azure Blob Storage:} Chmurowy magazyn danych do przechowywania plików multimedialnych (zdjęcia ogłoszeń, awatary).
    \end{itemize}
\end{itemize}

Komunikacja między klientami a serwerem odbywa się za pomocą protokołu HTTPS (dla REST API) oraz WebSocket (dla czatu w czasie rzeczywistym).

\begin{figure}[H]
    \centering
    \includegraphics[width=0.9\textwidth]{images/Architecture_Diagram.png}
    \caption{Ogólny diagram architektury systemu}
\end{figure}

\subsection{Projekt bazy danych}
Projekt wykorzystuje relacyjną bazę danych PostgreSQL. Schemat bazy danych został zaprojektowany w podejściu \textit{Code-First} przy użyciu JPA (Java Persistence API) i Hibernate. Poniżej znajduje się opis kluczowych tabel.

\begin{figure}[H]
    \centering
    \includegraphics[width=0.8\textwidth]{images/Diagram ERD.png}
    \caption{Diagram ERD}
\end{figure}

\subsubsection{Tabela \texttt{tbl\_users}}
Przechowuje informacje o użytkownikach systemu.
\begin{table}[H]
\centering
\begin{tabular}{|l|l|l|}
\hline
\textbf{Nazwa kolumny} & \textbf{Typ danych} & \textbf{Opis} \\
\hline
\texttt{id} & BIGINT (PK) & Klucz główny, auto-inkrementacja \\
\texttt{user\_id} & VARCHAR(255) & Publiczny, unikalny identyfikator (UUID) \\
\texttt{name} & VARCHAR(255) & Nazwa użytkownika \\
\texttt{email} & VARCHAR(255) & Unikalny adres email \\
\texttt{password} & VARCHAR(255) & Hasło (zahaszowane algorytmem BCrypt) \\
\texttt{profile\_image\_url} & VARCHAR(255) & URL do zdjęcia profilowego \\
\texttt{role} & VARCHAR(255) & Rola użytkownika (USER, ADMIN) \\
\texttt{is\_account\_verified} & BOOLEAN & Czy konto zostało zweryfikowane przez OTP \\
\texttt{verify\_otp} & VARCHAR(255) & Kod weryfikacyjny OTP \\
\texttt{verify\_otp\_expire\_at} & BIGINT & Czas wygaśnięcia kodu weryfikacyjnego \\
\texttt{reset\_otp} & VARCHAR(255) & Kod OTP do resetu hasła \\
\texttt{reset\_otp\_expire\_at} & BIGINT & Czas wygaśnięcia kodu resetu hasła \\
\texttt{created\_at} & TIMESTAMP & Data utworzenia konta \\
\texttt{updated\_at} & TIMESTAMP & Data ostatniej modyfikacji \\
\hline
\end{tabular}
\caption{Struktura tabeli \texttt{tbl\_users}}
\end{table}

\subsubsection{Tabela \texttt{listing}}
Główna tabela przechowująca ogłoszenia.
\begin{table}[H]
\centering
\begin{tabular}{|l|l|l|}
\hline
\textbf{Nazwa kolumny} & \textbf{Typ danych} & \textbf{Opis} \\
\hline
\texttt{id} & BIGINT (PK) & Klucz główny \\
\texttt{public\_id} & UUID & Publiczny, unikalny identyfikator \\
\texttt{seller\_user\_id} & VARCHAR(255) & ID sprzedawcy (FK do \texttt{tbl\_users}) \\
\texttt{category\_id} & BIGINT (FK) & ID kategorii (FK do \texttt{category}) \\
\texttt{title} & VARCHAR(255) & Tytuł ogłoszenia \\
\texttt{description} & TEXT & Opis ogłoszenia \\
\texttt{price\_amount} & NUMERIC(12, 2) & Cena \\
\texttt{currency} & VARCHAR(3) & Waluta (np. PLN) \\
\texttt{negotiable} & BOOLEAN & Czy cena podlega negocjacji \\
\texttt{location\_city} & VARCHAR(255) & Miasto \\
\texttt{location\_region} & VARCHAR(255) & Region/Województwo \\
\texttt{latitude} & DOUBLE PRECISION & Szerokość geograficzna \\
\texttt{longitude} & DOUBLE PRECISION & Długość geograficzna \\
\texttt{contact\_phone\_number} & VARCHAR(255) & Numer telefonu kontaktowego \\
\texttt{status} & VARCHAR(255) & Status ogłoszenia
(DRAFT,ACTIVE, etc.) \\
\texttt{published\_at} & TIMESTAMP & Data publikacji \\
\texttt{expires\_at} & TIMESTAMP & Data wygaśnięcia \\
\texttt{created\_at} & TIMESTAMP & Data utworzenia \\
\texttt{updated\_at} & TIMESTAMP & Data ostatniej modyfikacji \\
\hline
\end{tabular}
\caption{Struktura tabeli \texttt{listing}}
\end{table}

\subsubsection{Tabela \texttt{listing\_media}}
Przechowuje zdjęcia przypisane do ogłoszeń.
\begin{table}[H]
\centering
\begin{tabular}{|l|l|l|}
\hline
\textbf{Nazwa kolumny} & \textbf{Typ danych} & \textbf{Opis} \\
\hline
\texttt{id} & BIGINT (PK) & Klucz główny \\
\texttt{listing\_id} & BIGINT (FK) & ID ogłoszenia \\
\texttt{media\_url} & VARCHAR(255) & URL do pliku w chmurze \\
\texttt{media\_type} & VARCHAR(50) & Typ pliku (np. image) \\
\texttt{position} & INTEGER & Kolejność wyświetlania \\
\hline
\end{tabular}
\caption{Struktura tabeli \texttt{listing\_media}}
\end{table}

\subsubsection{Tabela \texttt{category}}
Przechowuje hierarchiczną strukturę kategorii.
\begin{table}[H]
\centering
\begin{tabular}{|l|l|l|}
\hline
\textbf{Nazwa kolumny} & \textbf{Typ danych} & \textbf{Opis} \\
\hline
\texttt{id} & BIGINT (PK) & Klucz główny \\
\texttt{parent\_id} & BIGINT (FK) & ID kategorii nadrzędnej (self-reference) \\
\texttt{name} & VARCHAR(255) & Nazwa kategorii \\
\texttt{sort\_order} & INTEGER & Kolejność sortowania \\
\texttt{image\_url} & VARCHAR(255) & Ikona/Obrazek kategorii \\
\texttt{depth} & INTEGER & Głębokość w drzewie kategorii \\
\texttt{is\_leaf} & BOOLEAN & Czy jest to kategoria końcowa (liść) \\
\hline
\end{tabular}
\caption{Struktura tabeli \texttt{category}}
\end{table}

\subsubsection{Tabela \texttt{attribute\_definition}}
Definiuje atrybuty (cechy), które mogą posiadać ogłoszenia w danej kategorii.
\begin{table}[H]
\centering
\begin{tabular}{|l|l|l|}
\hline
\textbf{Nazwa kolumny} & \textbf{Typ danych} & \textbf{Opis} \\
\hline
\texttt{id} & BIGINT (PK) & Klucz główny \\
\texttt{category\_id} & BIGINT (FK) & ID kategorii \\
\texttt{key} & VARCHAR(255) & Klucz atrybutu (np. color) \\
\texttt{label} & VARCHAR(255) & Etykieta wyświetlana (np. Kolor) \\
\texttt{type} & VARCHAR(50) & Typ danych (STRING,NUMBER,BOOLEAN,ENUM) \\
\texttt{unit} & VARCHAR(50) & Jednostka (np. km, cm) \\
\texttt{sort\_order} & INTEGER & Kolejność wyświetlania \\
\texttt{active} & BOOLEAN & Czy atrybut jest aktywny \\
\hline
\end{tabular}
\caption{Struktura tabeli \texttt{attribute\_definition}}
\end{table}

\subsubsection{Tabela \texttt{attribute\_option}}
Przechowuje możliwe wartości dla atrybutów typu ENUM.
\begin{table}[H]
\centering
\begin{tabular}{|l|l|l|}
\hline
\textbf{Nazwa kolumny} & \textbf{Typ danych} & \textbf{Opis} \\
\hline
\texttt{id} & BIGINT (PK) & Klucz główny \\
\texttt{attribute\_id} & BIGINT (FK) & ID definicji atrybutu \\
\texttt{value} & VARCHAR(255) & Wartość opcji \\
\texttt{label} & VARCHAR(255) & Etykieta opcji \\
\texttt{sort\_order} & INTEGER & Kolejność sortowania \\
\hline
\end{tabular}
\caption{Struktura tabeli \texttt{attribute\_option}}
\end{table}

\subsubsection{Tabela \texttt{listing\_attribute}}
Przechowuje konkretne wartości atrybutów dla danego ogłoszenia.
\begin{table}[H]
\centering
\begin{tabular}{|l|l|l|}
\hline
\textbf{Nazwa kolumny} & \textbf{Typ danych} & \textbf{Opis} \\
\hline
\texttt{id} & BIGINT (PK) & Klucz główny \\
\texttt{listing\_id} & BIGINT (FK) & ID ogłoszenia \\
\texttt{attribute\_id} & BIGINT (FK) & ID definicji atrybutu \\
\texttt{option\_id} & BIGINT (FK) & ID wybranej opcji (dla ENUM) \\
\texttt{v\_text} & VARCHAR(255) & Wartość tekstowa \\
\texttt{v\_number} & NUMERIC & Wartość liczbowa \\
\texttt{v\_boolean} & BOOLEAN & Wartość logiczna \\
\hline
\end{tabular}
\caption{Struktura tabeli \texttt{listing\_attribute}}
\end{table}

\subsubsection{Tabela \texttt{favorite}}
Przechowuje informacje o polubieniach (ulubionych ogłoszeniach).
\begin{table}[H]
\centering
\begin{tabular}{|l|l|l|}
\hline
\textbf{Nazwa kolumny} & \textbf{Typ danych} & \textbf{Opis} \\
\hline
\texttt{id} & BIGINT (PK) & Klucz główny \\
\texttt{user\_id} & VARCHAR(255) & ID użytkownika \\
\texttt{entity\_id} & VARCHAR(255) & ID polubionego obiektu (np. ogłoszenia) \\
\texttt{entity\_type} & VARCHAR(50) & Typ obiektu (np. LISTING) \\
\texttt{created\_at} & TIMESTAMP & Data dodania do ulubionych \\
\hline
\end{tabular}
\caption{Struktura tabeli \texttt{favorite}}
\end{table}

\subsubsection{Tabela \texttt{conversation}}
Reprezentuje konwersację między dwoma użytkownikami na temat ogłoszenia.
\begin{table}[H]
\centering
\begin{tabular}{|l|l|l|}
\hline
\textbf{Nazwa kolumny} & \textbf{Typ danych} & \textbf{Opis} \\
\hline
\texttt{id} & BIGINT (PK) & Klucz główny \\
\texttt{listing\_id} & UUID & ID ogłoszenia, którego dotyczy rozmowa \\
\texttt{buyer\_id} & VARCHAR(255) & ID kupującego \\
\texttt{seller\_id} & VARCHAR(255) & ID sprzedającego \\
\texttt{created\_at} & TIMESTAMP & Data utworzenia \\
\texttt{updated\_at} & TIMESTAMP & Data ostatniej wiadomości \\
\hline
\end{tabular}
\caption{Struktura tabeli \texttt{conversation}}
\end{table}

\subsubsection{Tabela \texttt{message}}
Przechowuje pojedynczą wiadomość w ramach konwersacji.
\begin{table}[H]
\centering
\begin{tabular}{|l|l|l|}
\hline
\textbf{Nazwa kolumny} & \textbf{Typ danych} & \textbf{Opis} \\
\hline
\texttt{id} & BIGINT (PK) & Klucz główny \\
\texttt{conversation\_id} & BIGINT (FK) & ID konwersacji \\
\texttt{sender\_id} & VARCHAR(255) & ID nadawcy \\
\texttt{recipient\_id} & VARCHAR(255) & ID odbiorcy \\
\texttt{content} & TEXT & Treść wiadomości \\
\texttt{is\_read} & BOOLEAN & Czy wiadomość została odczytana \\
\texttt{created\_at} & TIMESTAMP & Data wysłania \\
\hline
\end{tabular}
\caption{Struktura tabeli \texttt{message}}
\end{table}

\subsection{Szczegółowa dokumentacja kodu (Backend)}
Część serwerowa aplikacji została zaimplementowana w języku Java (wersja 21) przy użyciu frameworka Spring Boot. Architektura projektu opiera się na podziale na moduły funkcjonalne , co zapewnia niskie sprzężenie komponentów.

\subsubsection{Architektura i Pakiety}
Kod źródłowy został podzielony na niezależne pakiety funkcjonalne. Poniżej przedstawiono opis najważniejszych z nich.

\paragraph{Pakiet \texttt{com.BINM.user}}
Ten moduł odpowiada za zarządzanie tożsamością użytkowników, autentykację oraz obsługę profili.

\textbf{Główne odpowiedzialności:}
\begin{itemize}
    \item Rejestracja i logowanie użytkowników (generowanie tokenów JWT).
    \item Weryfikacja konta (OTP) i resetowanie haseł.
    \item Zarządzanie danymi profilowymi (imię, zdjęcie).
    \item Konfiguracja Spring Security (filtry, role).
\end{itemize}

\textbf{Struktura pakietu:}
\begin{itemize}
    \item \texttt{controller}:
        \begin{itemize}
            \item \texttt{AuthController}: Obsługuje publiczne endpointy związane z autentykacją (logowanie, rejestracja, reset hasła).
            \item \texttt{ProfileController}: Obsługuje endpointy wymagające zalogowania, związane z zarządzaniem własnym profilem (pobieranie danych, edycja, weryfikacja OTP).
            \item \texttt{PublicUserController}: Obsługuje publiczne endpointy do pobierania danych o innych użytkownikach (np. sprzedawcach).
        \end{itemize}
    \item \texttt{service}:
        \begin{itemize}
            \item \texttt{ProfileFacade}: Interfejs definiujący kontrakt dla operacji na profilach, ukrywający implementację.
            \item \texttt{ProfileService}: Implementacja logiki biznesowej (tworzenie konta, walidacja, wysyłanie emaili).
            \item \texttt{AppUserDetailsService}: Implementacja interfejsu Spring Security \texttt{UserDetailsService}, ładująca użytkownika z bazy danych.
            \item \texttt{CustomUserDetails}: Rozszerzenie klasy \texttt{User} ze Spring Security, przechowujące dodatkowe informacje (np. ID użytkownika).
        \end{itemize}
    \item \texttt{repository}:
        \begin{itemize}
            \item \texttt{UserRepository}: Interfejs Spring Data JPA do komunikacji z tabelą \texttt{tbl\_users}.
        \end{itemize}
    \item \texttt{model}:
        \begin{itemize}
            \item \texttt{UserEntity}: Encja JPA reprezentująca użytkownika w bazie danych.
            \item \texttt{UserRole}: Enum definiujący role w systemie (USER, ADMIN).
        \end{itemize}
    \item \texttt{io}: Obiekty DTO (Data Transfer Object) służące do przesyłania danych między klientem a serwerem (np. \texttt{AuthRequest}, \texttt{ProfileResponse}).
    \item \texttt{util}:
        \begin{itemize}
            \item \texttt{JwtUtil}: Klasa narzędziowa odpowiedzialna za generowanie, podpisywanie i walidację tokenów JWT.
        \end{itemize}
    \item \texttt{config}:
        \begin{itemize}
            \item \texttt{SecurityConfig}: Główna konfiguracja bezpieczeństwa (CORS, CSRF, reguły dostępu do endpointów).
            \item \texttt{JwtRequestFilter}: Filtr przechwytujący każde żądanie HTTP, weryfikujący token JWT i ustawiający kontekst bezpieczeństwa.
            \item \texttt{AdminSeeder}: Komponent uruchamiany przy starcie aplikacji, tworzący domyślne konto administratora, jeśli nie istnieje.
        \end{itemize}
\end{itemize}

\begin{figure}[H]
    \centering
    \includegraphics[width=1\textwidth]{images/uml_user_package.png}
    \caption{Uproszczony diagram klas pakietu \texttt{com.BINM.user}}
\end{figure}

\paragraph{Pakiet \texttt{com.BINM.listing}}
Jest to największy i najważniejszy moduł aplikacji, odpowiedzialny za całą logikę związaną z ogłoszeniami, kategoriami i atrybutami.

\textbf{Główne odpowiedzialności:}
\begin{itemize}
    \item Zarządzanie cyklem życia ogłoszenia (DRAFT, WAITING, ACTIVE, EXPIRED, etc.).
    \item Obsługa zaawansowanego wyszukiwania i filtrowania.
    \item Zarządzanie hierarchiczną strukturą kategorii.
    \item Zarządzanie dynamicznymi atrybutami przypisanymi do kategorii.
\end{itemize}

\textbf{Struktura pakietu:}
\begin{itemize}
    \item \texttt{listing}:
        \begin{itemize}
            \item \texttt{ListingPublicController}, \texttt{ListingUserController}, \texttt{ListingAdminController}: Kontrolery obsługujące operacje na ogłoszeniach dla różnych ról.
            \item \texttt{ListingService}: Główny serwis implementujący logikę biznesową ogłoszeń.
            \item \texttt{SearchService}: Serwis dedykowany do wyszukiwania.
            \item \texttt{ListingValidator}: Komponent walidujący reguły biznesowe.
            \item \texttt{ListingRepository}, \texttt{ListingMediaRepository}, \texttt{ListingAttributeRepository}: Repozytoria JPA.
            \item \texttt{ListingExceptionHandler}: Obsługa błędów specyficznych dla ogłoszeń.
            \item \texttt{ListingMapper}: Mapper MapStruct konwertujący obiekty między warstwami (Entity $\leftrightarrow$ DTO).
            \item \texttt{ListingScheduler}: Komponent cykliczny zarządzający wygaszaniem ogłoszeń.
            \item \texttt{ListingSeeder}: Klasa zasilająca bazę przykładowymi ogłoszeniami.
        \end{itemize}
    \item \texttt{category}:
        \begin{itemize}
            \item \texttt{CategoryPublicController}, \texttt{CategoryAdminController}: Zarządzanie kategoriami.
            \item \texttt{CategoryService}: Logika drzewa kategorii.
            \item \texttt{CategoryMapper}: Mapper MapStruct dla kategorii.
            \item \texttt{CategorySeeder}: Inicjalizacja drzewa kategorii przy starcie aplikacji.
            \item \texttt{CategoryCacheWarmup}: Komponent odpowiedzialny za wstępne załadowanie struktury kategorii do pamięci podręcznej (Cache) w celu optymalizacji wydajności.
        \end{itemize}
    \item \texttt{attribute}:
        \begin{itemize}
            \item \texttt{AttributeAdminController}: Zarządzanie definicjami atrybutów.
            \item \texttt{AttributeService}: Logika biznesowa atrybutów.
            \item \texttt{AttributeDefinitionRepository}, \texttt{AttributeOptionRepository}: Repozytoria JPA.
            \item \texttt{AttributeMapper}: Mapper MapStruct dla atrybutów.
            \item \texttt{AttributeSeeder}: Inicjalizacja definicji atrybutów.
        \end{itemize}
\end{itemize}

\begin{figure}[H]
    \centering
    \includegraphics[width=0.8\textwidth]{images/uml_listing_package.png}
    \caption{Uproszczony diagram klas pakietu \texttt{com.BINM.listing}}
\end{figure}

\paragraph{Pakiet \texttt{com.BINM.messaging}}
Moduł odpowiedzialny za komunikację w czasie rzeczywistym między użytkownikami.

\textbf{Główne odpowiedzialności:}
\begin{itemize}
    \item Obsługa połączeń WebSocket (STOMP).
    \item Przesyłanie wiadomości w czasie rzeczywistym.
    \item Zapisywanie historii konwersacji w bazie danych.
    \item Zarządzanie statusem wiadomości (przeczytana/nieprzeczytana).
\end{itemize}

\textbf{Struktura pakietu:}
\begin{itemize}
    \item \texttt{controller}:
        \begin{itemize}
            \item \texttt{ChatController}: Kontroler WebSocket obsługujący wysyłanie wiadomości.
            \item \texttt{ConversationController}: Kontroler REST do pobierania historii rozmów.
        \end{itemize}
    \item \texttt{service}:
        \begin{itemize}
            \item \texttt{MessagingFacade}: Interfejs publiczny modułu.
            \item \texttt{MessagingService}: Logika biznesowa (tworzenie konwersacji, zapis wiadomości).
        \end{itemize}
    \item \texttt{repository}:
        \begin{itemize}
            \item \texttt{ConversationRepository}, \texttt{MessageRepository}: Dostęp do bazy danych.
        \end{itemize}
    \item \texttt{config}:
        \begin{itemize}
            \item \texttt{WebSocketConfig}: Konfiguracja brokera wiadomości i endpointów STOMP.
            \item \texttt{JwtChannelInterceptor}: Interceptor autoryzujący połączenia WebSocket za pomocą tokenu JWT.
            \item \texttt{WebSocketUserPrincipal}: Klasa reprezentująca tożsamość użytkownika w sesji WebSocket.
        \end{itemize}
\end{itemize}

\begin{figure}[H]
    \centering
    \includegraphics[width=0.8\textwidth]{images/uml_messaging_package.png}
    \caption{Uproszczony diagram klas pakietu \texttt{com.BINM.messaging}}
\end{figure}

\paragraph{Pakiet \texttt{com.BINM.interactions}}
Moduł obsługujący interakcje użytkowników z ogłoszeniami.

\textbf{Główne odpowiedzialności:}
\begin{itemize}
    \item Dodawanie i usuwanie ogłoszeń z ulubionych.
    \item Sprawdzanie statusu polubienia.
    \item Pobieranie listy ulubionych ogłoszeń użytkownika.
\end{itemize}

\textbf{Struktura pakietu:}
\begin{itemize}
    \item \texttt{controller}:
        \begin{itemize}
            \item \texttt{InteractionController}: Obsługuje endpointy REST do zarządzania ulubionymi.
        \end{itemize}
    \item \texttt{service}:
        \begin{itemize}
            \item \texttt{InteractionFacade}: Interfejs publiczny modułu.
            \item \texttt{InteractionService}: Implementacja logiki biznesowej. Nasłuchuje również na zdarzenia (np. zakończenie ogłoszenia), aby usunąć nieaktualne polubienia.
        \end{itemize}
    \item \texttt{repository}:
        \begin{itemize}
            \item \texttt{FavoriteRepository}: Dostęp do tabeli \texttt{favorite}.
        \end{itemize}
    \item \texttt{model}:
        \begin{itemize}
            \item \texttt{Favorite}: Encja reprezentująca polubienie.
            \item \texttt{EntityType}: Enum określający typ polubionego obiektu.
        \end{itemize}
\end{itemize}

\begin{figure}[H]
    \centering
    \includegraphics[width=0.8\textwidth]{images/uml_interactions_package.png}
    \caption{Uproszczony diagram klas pakietu \texttt{com.BINM.interactions}}
\end{figure}

\paragraph{Pakiet \texttt{com.BINM.media}}
Moduł odpowiedzialny za obsługę plików multimedialnych (zdjęcia ogłoszeń, zdjęcia profilowe).

\textbf{Główne odpowiedzialności:}
\begin{itemize}
    \item Przesyłanie plików na zewnętrzny serwer (Azure Blob Storage).
    \item Generowanie publicznych adresów URL do zasobów.
\end{itemize}

\textbf{Struktura pakietu:}
\begin{itemize}
    \item \texttt{controller}:
        \begin{itemize}
            \item \texttt{MediaController}: Endpointy do uploadu plików.
        \end{itemize}
    \item \texttt{service}:
        \begin{itemize}
            \item \texttt{MediaFacade}: Interfejs publiczny modułu.
            \item \texttt{MediaService}: Implementacja logiki uploadu.
        \end{itemize}
    \item \texttt{storage}:
        \begin{itemize}
            \item \texttt{ImageStorageClient}: Interfejs definiujący kontrakt dla operacji na plikach.
            \item \texttt{ImageStorageClientImpl}: Implementacja klienta integrującego się z Azure Blob Storage SDK.
        \end{itemize}
\end{itemize}

\begin{figure}[H]
    \centering
    \includegraphics[width=0.8\textwidth]{images/uml_media_package.png}
    \caption{Uproszczony diagram klas pakietu \texttt{com.BINM.media}}
\end{figure}

\paragraph{Pakiet \texttt{com.BINM.mailing}}
Moduł pomocniczy odpowiedzialny za wysyłanie wiadomości email (np. kody OTP).

\textbf{Główne odpowiedzialności:}
\begin{itemize}
    \item Wysyłanie emaili z kodami weryfikacyjnymi.
    \item Integracja z serwerem SMTP.
\end{itemize}

\textbf{Struktura pakietu:}
\begin{itemize}
    \item \texttt{service}:
        \begin{itemize}
            \item \texttt{EmailFacade}: Interfejs publiczny modułu.
            \item \texttt{EmailService}: Implementacja wykorzystująca \texttt{JavaMailSender}.
        \end{itemize}
\end{itemize}

\subsubsection{Kluczowe Klasy i Interfejsy}

\textbf{1. Fasady}
Projekt wykorzystuje wzorzec Fasady, aby ukryć złożoność logiki biznesowej przed kontrolerami i innymi modułami.
\begin{itemize}
    \item \texttt{ListingFacade}: Główny punkt dostępu do operacji na ogłoszeniach (tworzenie, edycja, wyszukiwanie).
    \item \texttt{ProfileFacade}: Zarządzanie profilem użytkownika i autentykacją.
    \item \texttt{MessagingFacade}: Obsługa wiadomości i konwersacji.
    \item \texttt{InteractionFacade}: Zarządzanie ulubionymi.
    \item \texttt{SearchFacade}: Dedykowany interfejs do wyszukiwania ogłoszeń.
    \item \texttt{MediaFacade}: Obsługa operacji na plikach multimedialnych.
    \item \texttt{EmailFacade}: Obsługa wysyłki wiadomości email.
\end{itemize}

\textbf{2. Serwisy}
Implementują logikę biznesową zdefiniowaną w interfejsach fasad.
\begin{itemize}
    \item \texttt{ListingService}: Odpowiada za cykl życia ogłoszenia (Draft $\rightarrow$ Waiting $\rightarrow$ Active $\rightarrow$ Completed/Expired). Implementuje logikę walidacji i zapisu.
    \item \texttt{SearchService}: Dedykowany serwis do obsługi zaawansowanego wyszukiwania ogłoszeń z wykorzystaniem dynamicznych filtrów.
    \item \texttt{MessagingService}: Obsługuje logikę czatu, tworzenie konwersacji i zapisywanie wiadomości.
    \item \texttt{InteractionService}: Obsługuje logikę ulubionych.
    \item \texttt{MediaService}: Obsługuje logikę przesyłania plików do chmury.
    \item \texttt{EmailService}: Obsługuje wysyłkę emaili przez SMTP.
    \item \texttt{ProfileService}: Obsługuje logikę użytkownika (rejestracja, logowanie, profil).
\end{itemize}

\textbf{3. Kontrolery}
Wystawiają endpointy REST API. Podzielone są na publiczne (dostępne dla wszystkich), użytkownika (wymagające logowania) i administracyjne (wymagające roli ADMIN).
\begin{itemize}
    \item \texttt{ListingPublicController}: Publiczne endpointy do przeglądania i wyszukiwania ogłoszeń.
    \item \texttt{ListingUserController}: Endpointy dla zalogowanych użytkowników (zarządzanie własnymi ogłoszeniami).
    \item \texttt{ListingAdminController}: Endpointy administracyjne (moderacja ogłoszeń).
    \item \texttt{CategoryPublicController}: Publiczne endpointy do pobierania kategorii.
    \item \texttt{CategoryAdminController}: Zarządzanie kategoriami (ADMIN).
    \item \texttt{AttributeAdminController}: Zarządzanie atrybutami (ADMIN).
    \item \texttt{AuthController}: Logowanie, rejestracja, reset hasła.
    \item \texttt{ProfileController}: Zarządzanie własnym profilem.
    \item \texttt{PublicUserController}: Pobieranie publicznych profili innych użytkowników.
    \item \texttt{ChatController}: Kontroler WebSocket.
    \item \texttt{ConversationController}: Historia rozmów (REST).
    \item \texttt{InteractionController}: Zarządzanie ulubionymi.
    \item \texttt{MediaController}: Upload plików.
\end{itemize}

\textbf{4. Konfiguracja i Bezpieczeństwo}
Kluczowe klasy odpowiedzialne za konfigurację aplikacji i bezpieczeństwo.
\begin{itemize}
    \item \texttt{SecurityConfig}: Główna klasa konfiguracyjna Spring Security. Definiuje łańcuch filtrów bezpieczeństwa, reguły dostępu do endpointów (np. \texttt{/admin/**} tylko dla roli ADMIN) oraz konfigurację CORS.
    \item \texttt{JwtRequestFilter}: Filtr uruchamiany przed każdym żądaniem HTTP. Weryfikuje poprawność tokenu JWT przesyłanego w nagłówku \texttt{Authorization}, parsuje go i ustawia kontekst bezpieczeństwa (\texttt{SecurityContextHolder}).
    \item \texttt{CustomAuthenticationEntryPoint}: Obsługuje błędy autentykacji (np. próba dostępu do zasobu bez tokenu), zwracając odpowiedni kod błędu 401.
    \item \texttt{WebSocketConfig}: Konfiguruje brokera wiadomości STOMP oraz endpointy WebSocket. Rejestruje interceptory.
    \item \texttt{JwtChannelInterceptor}: Specjalny interceptor dla kanałów WebSocket, który autoryzuje połączenie na podstawie tokenu JWT w momencie nawiązywania sesji.
\end{itemize}

\subsubsection{Wzorce Projektowe}
W projekcie zastosowano szereg wzorców projektowych, które zapewniają modularność, skalowalność i łatwość utrzymania kodu.

\begin{itemize}
    \item \textbf{Model-View-Controller (MVC):} Architektoniczny wzorzec projektowy, który jest fundamentem frameworka Spring Web.
    \begin{itemize}
        \item \textbf{Model:} Reprezentowany przez encje JPA (\texttt{UserEntity}, \texttt{Listing}) oraz obiekty DTO. Przechowuje dane i logikę biznesową.
        \item \textbf{View:} W przypadku REST API widokiem jest format JSON zwracany do klienta.
        \item \textbf{Controller:} Klasy z adnotacją \texttt{@RestController} (np. \texttt{ListingPublicController}), które przyjmują żądania HTTP, przetwarzają je (często delegując do serwisu) i zwracają odpowiedź.
    \end{itemize}
    \begin{figure}[H]
        \centering
        \includegraphics[width=1.0\textwidth]{images/MVC.png}
        \caption{Przykładowy przebieg informacji w aplikacji.}
    \end{figure}
    \item \textbf{Fasada:} Używana do uproszczenia interfejsów modułów i ukrycia szczegółów implementacyjnych.
    \begin{itemize}
        \item \textbf{Zastosowanie:} Interfejsy takie jak \texttt{ListingFacade} czy \texttt{ProfileFacade} stanowią jedyny punkt wejścia do logiki danego modułu. Dzięki temu kontrolery nie muszą znać wewnętrznych zależności serwisu (np. repozytoriów czy mapperów). Ułatwia to testowanie i refaktoryzację.
    \end{itemize}

    \item \textbf{Strategia:} Wzorzec behawioralny, który pozwala na zdefiniowanie rodziny algorytmów i ich wymienne stosowanie.
    \begin{itemize}
        \item \textbf{Zastosowanie:} Wykorzystywany w mechanizmie wyszukiwania (\texttt{Specification <Listing>}). Kryteria filtrowania (np. po cenie, kategorii, atrybutach) są budowane dynamicznie w zależności od żądania użytkownika. Każdy filtr jest osobną strategią (predykatem), które są łączone w jedno zapytanie.
    \end{itemize}
    \begin{figure}[H]
        \centering
        \includegraphics[width=1.0\textwidth]{images/STRATEGY.png}
        \caption{Przykład wykorzystania interfejsu Specyfication do budowania strategi wyszukiwania.}
    \end{figure}
    \item \textbf{Obserwator:} Wzorzec, w którym obiekt (podmiot) powiadamia listę zależnych od niego obiektów (obserwatorów) o zmianie swojego stanu.
    \begin{itemize}
        \item \textbf{Zastosowanie:} Zrealizowany za pomocą mechanizmu Spring Events. Przykład: Gdy ogłoszenie zostaje zakończone w \texttt{ListingService}, publikowane jest zdarzenie \texttt{ListingFinishedEvent}. Serwis \texttt{InteractionService} nasłuchuje na to zdarzenie (\texttt{@EventListener}) i automatycznie usuwa to ogłoszenie z ulubionych. Dzięki temu moduły są luźno powiązane.}.
    \end{itemize}

    \item \textbf{Budowniczy:} Wzorzec konstrukcyjny, który pozwala na tworzenie złożonych obiektów krok po kroku.
    \begin{itemize}
        \item \textbf{Zastosowanie:} Wykorzystywany powszechnie dzięki bibliotece Lombok (\texttt{@Builder}). Umożliwia czytelne tworzenie obiektów DTO i encji, np.:
        \begin{verbatim}
        UserEntity.builder()
            .email("...")
            .name("...")
            .build();
        \end{verbatim}
    \end{itemize}

    \item \textbf{Data Transfer Object (DTO):} Wzorzec służący do przesyłania danych między podsystemami aplikacji.
    \begin{itemize}
        \item \textbf{Zastosowanie:} Obiekty w pakietach \texttt{dto} lub \texttt{io} (np. \texttt{ListingDto}, \texttt{ProfileRequest}) służą do oddzielenia wewnętrznego modelu danych (encji JPA) od danych prezentowanych klientowi API. Zapobiega to wyciekowi wrażliwych danych i problemom z cyklicznymi zależnościami w JSON.
    \end{itemize}

    \item \textbf{Wstrzykiwanie Zależności:} Wzorzec, w którym obiekty nie tworzą swoich zależności, ale otrzymują je z zewnątrz.
    \begin{itemize}
        \item \textbf{Zastosowanie:} Fundament frameworka Spring. Wszystkie serwisy, kontrolery i repozytoria są wstrzykiwane przez konstruktor
        (z użyciem \texttt{@RequiredArgsConstructor}), co ułatwia testowanie i zarządzanie cyklem życia komponentów.
    \end{itemize}
\end{itemize}

\subsection{Szczegółowa dokumentacja kodu (Frontend)}
\textit{[Miejsce dla zespołu frontendowego]}

\subsection{Szczegółowa dokumentacja kodu (Mobile)}
\textit{[Miejsce dla zespołu mobilnego]}

\subsection{Środowisko programistyczne}
Do rozwoju i uruchomienia projektu wykorzystano następujące narzędzia i technologie:

\begin{itemize}
    \item \textbf{Język programowania:} Java 21.
    \item \textbf{Framework:} Spring Boot 3.x (Spring Web, Spring Security, Spring Data JPA, Spring WebSocket).
    \item \textbf{Baza danych:} PostgreSQL 15+.
    \item \textbf{Zarządzanie zależnościami:} Apache Maven.
    \item \textbf{IDE:} IntelliJ IDEA Ultimate / Community.
    \item \textbf{Konteneryzacja:} Docker (do uruchomienia bazy danych).
    \item \textbf{Przechowywanie plików:} Azure Blob Storage.
    \item \textbf{Biblioteki pomocnicze:} Lombok (redukcja boilerplate code), MapStruct (mapowanie DTO-Entity), JJWT (obsługa tokenów JWT).
\end{itemize}





\renewcommand\refname{Literatura (jeżeli wymagana)}
\bibliography{references}
\addcontentsline{toc}{section}{Literatura}
% --------------------------------------------------------------------
%%%%%%% odkomentować gdy bibliografia ma być wewnątrz dokumentu
% --------------------------------------------------------------------
%\begin{thebibliography}{11}
%
%\addcontentsline{toc}{section}{Literatura}
%
%\bibitem{ZAN}
%C. Zannoni and P. Pasini,
%\emph{Advances in the Computer Simulatons of Liquid Crystals}, Kluwer Academic Publishers, 2000.
%
%\end{thebibliography}

\end{document}
